\documentclass[12pt,letterpaper]{article}

\usepackage[utf8]{inputenc}
\usepackage[T1]{fontenc}
\usepackage[english]{babel}
\usepackage{listings, float, enumitem}
\usepackage{graphicx, geometry, xcolor}
\usepackage[normalem]{ulem}
\usepackage[colorlinks=true, linkcolor=linkblue, urlcolor=linkblue, citecolor=linkblue]{hyperref}

\geometry{margin=3cm, top=3cm, bottom=3cm}

\begin{document}

\begin{titlepage}
  \thispagestyle{empty}
  %--------------- COMANDO LINEA----------------->
  \newcommand{\linea}{\rule{\linewidth}{0.5mm}}                 
  \center
  %<<<<<<<<<<<<<<<<<<<<<<<<<<<<<<<<<<<<<<<<<<<

  %Add logos
  \includegraphics[width=0.24\textwidth]{~/unam_logo.png} \hspace{1.5cm}
  \includegraphics[width=0.25\textwidth]{~/fc_logo.png} \\[1cm]
  \textbf{\Large UNIVERSIDAD NACIONAL AUTÓNOMA}\\[3mm]
  \textbf{\Large DE MÉXICO} \\[6mm]
  \textit{\Large Facultad de Ciencias}

  \vfill

  %---------------TÍTULO----------------->
  \linea
  \vfill \vspace{1cm}
  \textbf{\Large Inteligencia Artificial}\\[1.1cm]
  \textbf{\Large Búsqueda, Búsqueda Adversarial y Local}\\[0.9cm]
  \linea \\
  \vfill
  \vspace{0.7cm}
  %Title of the Research
  \textbf{\Large  Tarea 1}\\[6mm]
  %<<<<<<<<<<<<<<<<<<<<<<<<<<<<<<<<<<<<<<<<<<<

  %Team
  \vfill
  \textit{\small Presenta:}\\
  \textbf{\large  \textbf{Reyes Colín Luis Manuel}}\\[3mm]
  \textbf{\large  \textbf{Rojo Peña Manuel Ianluck}}\\[3mm]

  %Profesor y Grupo
  \textit{\small Profesor:}\\
  \textbf{Víctor Germán Mijangos de la Cruz}\\[3mm]
  \small Grupo: 7001, 2026-2\\[3mm]
  \vfill

  %Fecha or Date
  \textit{\small Fecha de entrega:}\\
         {\large 00 de febrero, 2026}
         
\end{titlepage}


\newpage

\begin{enumerate}[label=\arabic*.]
\item Considérese el siguiente mapa, donde el agente es el cuadrado azul (estado inicial) y la meta el verde (estado final):
  \begin{center}
    \includegraphics[width=0.4\textwidth]{f1.jpg}
  \end{center}
  Expresar de manera formal el problema como $SP = (A,S,s_0,F,T,c)$, considerar sólo los espacios blancos como estados de movimiento posibles y sólo considerar las acciones legales. Expresa las transiciones como una función con las acciones como parámetros.
  
\item Del ejercicio 1, ejecuta el algoritmo de primero en profundidad aplicando las acciones en el siguiente orden: Arriba, Abajo, Izquierda, Derecha. Usa Early Goal Test. Dibuja el árbol y la pila.
  
\item Del ejercicio 1, aplicar el algoritmo de primero en amplitud para obtener una solución utilizando Early Goal Test. Dibujar el árbol de búsqueda y la pila.
  
\item A partir del problema 1, calcular la heurística de distancia Manhattan para todos los estados y usar el algoritmo de primero mejor ambicioso con esta heurística para encontrar la solución.
  
\item Utilizar el algoritmo de Dijkstra para encontrar una solución al ejercicio 1. Dibujar el árbol de búsqueda y la pila. ¿Cuál es el costo del camino que resulta?
  
\item Utilizar la siguiente heurística para aplicar al ejercicio anterior el algoritmo de A$^*$ (dibujar árbol y pila):
  \begin{center}
    \begin{tabular}{c|ccccccccc}
      & $s_0$ & $s_1$ & $s_2$ & $s_3$ & $s_4$ & $s_5$ & $s_6$ & $s_7$ & $s_8$ \\
      \hline
      $h$ & $4$ & $15$ & $1$ & $0$ & $1$ & $0$ & $20$ & $35$ & $\infty$
    \end{tabular}
  \end{center}
  ¿Cuál es el costo del camino y cuál es la heurística final de éste?
  
\item Considerar la siguiente gráfica de un problema de búsqueda ($s_0$ es inicial y $s_7$ final):
  \begin{center}
    \includegraphics[width=0.7\textwidth]{f2.jpg}
  \end{center}
  \begin{enumerate}
  \item[a)] Expresar formalmente, la forma del problema de búsqueda $SP = (A,S,s_0,F,T,c)$.
  \item[b)] Utilizar el algoritmo de primero en amplitud para encontrar una solución. No expandir los nodos alcanzados ya. Usar Early Goal Test. Dibujar el árbol y la frontera.
  \end{enumerate}
  \begin{center}
    \begin{tabular}{c|cccccccc}
      & $s_0$ & $s_1$ & $s_2$ & $s_3$ & $s_4$ & $s_5$ & $s_6$ & $s_7$ \\
      \hline
      $h$ & $3$ & $2$ & $3$ & $2$ & $1$ & $2$ & $0$ & $0$
    \end{tabular}
  \end{center}
  \begin{enumerate}
  \item[c)] Utilizar el algoritmo de Greedy Best First Search para calcular una solución al problema. Dibujar el árbol y la pila.
  \item[d)] Utilizar la heurística anterior para encontrar una solución al problema utilizando el algoritmo A$^*$. No considerar los nodos ya alcanzados (una vez expandidos no se vuelven a expandir). Dibujar el árbol y la pila.
  \end{enumerate}
  
\item Utilizar el algoritmo de Beam Search con $k=2$ para encontrar una solución al problema del ejercicio 7. ¿Cuáles son los caminos generados?

\item Demostrar que si $h(s_1),\, s \in S$, es una heurística admisible para el algoritmo de A$^*$, entonces el algoritmo de A$^*$ encontrará una solución óptima.
  
\item Demostrar que el algoritmo de $A^*$ es completo siempre que exista una solución en el problema.

\item Dado el siguiente estado de un juego de gato:
  \begin{center}
\begin{verbatim}
x | x |  
o | o | x
  |   | o
\end{verbatim}
  \end{center}

Con ‘o’ siendo jugador MAX($1$) y ‘x’ siendo MIN($-1$) y el siguiente jugador en tirar es MAX. Determinar cuál es la siguiente mejor jugada para cada jugador y dibujar el árbol de juego.

\item Utiliza el método de $\alpha - \beta$ en el problema anterior.

\item Dada las siguientes cadenas de población inicial:
  \begin{itemize}
  \item $000000$
  \item $111010$
  \item $010100$
  \item $000001$
  \end{itemize}
  
  Usar el algoritmo genético para obtener un candidato final con mayor valor de fitness. Considerar los siguientes puntos:
  
  \begin{enumerate}
  \item[a)] Tomar en cada iteración los $3$ candidatos más altos.
  \item[b)] Combinar los dos primeros candidatos con más fitness (1 y 2), y los dos úlitmos (2 y 3). Si tienen el mismo fitness se combinarán entre ellos.
  \item[c)] Tomar recombinación ordenada por mitades de candidatos (mitad del primero más mitad del segundo).
  \item[d)] Repetir por 3 iteraciones.
  \end{enumerate}
  
\end{enumerate}
\end{document}
